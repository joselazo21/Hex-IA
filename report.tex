\documentclass[a4paper,12pt]{article}
\usepackage[utf8]{inputenc}
\usepackage[T1]{fontenc}
\usepackage{amsmath}
\usepackage{amsfonts}
\usepackage{geometry}
\geometry{margin=2.5cm}
\usepackage{enumitem}
\usepackage{titling}

\title{Jugador Inteligente de Hex: Implementación y Heurística}
\author{José Ernesto Morales Lazo}
\date{}

\begin{document}

\maketitle

\section{Introducción}
Este informe describe la implementación de un jugador inteligente para el juego de Hex, basado en el algoritmo Minimax con poda alfa-beta para optimizar la eficiencia computacional. Se detalla una heurística diseñada para evaluar estados del tablero, enfocada en maximizar la conectividad del jugador mientras considera el bloqueo estratégico del oponente.

\section{Implementación del Jugador Inteligente}
La implementación del jugador de Hex priorizó la eficiencia mediante las siguientes técnicas:

\begin{enumerate}
    \item \textbf{Poda Alfa-Beta}: Reduce el espacio de búsqueda en el algoritmo Minimax al descartar ramas del árbol de juego que no afectan la decisión final, optimizando el tiempo de cómputo.
    \item \textbf{Heurística Estratégica}: Evalúa estados no terminales para guiar la selección de movimientos, maximizando la conectividad del jugador y considerando la necesidad de bloquear las rutas del oponente.
\end{enumerate}

\section{Explicación de la Heurística}
La heurística implementada en la función \texttt{evaluate\_position} combina tres componentes para estimar la calidad de un estado del tablero:

\subsection{Costo Estimado del Jugador}
Este componente estima el número mínimo de movimientos necesarios para que el jugador conecte sus lados designados (izquierda-derecha para el jugador 1, arriba-abajo para el jugador 2). Se calcula mediante:

\begin{itemize}
    \item \textbf{Identificación de Componentes Conexas}: Utiliza las funciones \texttt{get\_connected\_components} y \texttt{flood\_fill} para detectar grupos de celdas conectadas del jugador.
    \item \textbf{Cálculo del Camino Más Corto}: Emplea búsqueda en anchura (BFS) para encontrar la distancia mínima desde componentes que tocan el lado inicial hasta el lado final.
    \item \textbf{Caso Especial}: Si no hay componentes que toquen el lado inicial, se estima el camino directamente desde el lado inicial.
\end{itemize}

Este enfoque guía al algoritmo hacia movimientos que construyan o extiendan caminos hacia el objetivo de conexión.

\subsection{Costo Estimado del Oponente}
Similar al costo del jugador, este componente estima los movimientos necesarios para que el oponente conecte sus lados, utilizando el mismo procedimiento basado en componentes conexas y BFS. Se pondera por 0.8 para priorizar ligeramente el progreso del jugador sobre el bloqueo del oponente. Este factor es crucial, ya que una conexión del oponente resulta en una derrota inmediata.

\subsection{Ventaja de Control del Centro}
Mide el control del jugador sobre la región central del tablero, definida como un círculo de radio $\lfloor \text{tamaño del tablero} / 3 \rfloor$ alrededor del centro. Se calcula la diferencia entre las piezas del jugador y del oponente en esta región, normalizada por el total de celdas y escalada por 0.01 para limitar su influencia. Controlar el centro proporciona flexibilidad estratégica, ya que facilita la construcción de múltiples caminos.

\subsection{Fórmula de la Heurística}
La puntuación heurística se define como:
\[
\text{Puntuación} = \text{costo estimado del jugador} - 0.8 \cdot \text{costo estimado del oponente} + 0.01 \cdot \text{ventaja de control del centro}
\]
Donde:
\begin{itemize}
    \item Un \emph{costo estimado del jugador} menor indica que el jugador está más cerca de conectar sus lados, aumentando la puntuación.
    \item Un \emph{costo estimado del oponente} mayor implica que el oponente está más lejos de ganar, incrementando la puntuación.
    \item Una \emph{ventaja de control del centro} positiva favorece posiciones con mayor control central, ofreciendo flexibilidad estratégica.
\end{itemize}

\end{document}